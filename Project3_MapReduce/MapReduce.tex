\documentclass{article} % For LaTeX2e
\usepackage{nips13submit_e,times}
%\usepackage{hyperref}
\usepackage[hyphens]{url}
\usepackage{alltt}
\usepackage{multirow, caption}
\usepackage{algorithm}
\usepackage{amsmath}
\usepackage{graphicx}
\usepackage[noend]{algpseudocode}
\usepackage{etoolbox}
\usepackage{hyperref}
\usepackage{fancyvrb}
\usepackage{tgcursor}
\usepackage{amsfonts}
\usepackage[ampersand]{easylist}
\newcommand\userinput[1]{\textbf{#1}}
\newcommand\comment[1]{\textit{#1}}
\newcommand\stdout[1]{\textsl{#1}}

\makeatletter
\preto{\@verbatim}{\topsep=0pt \partopsep=0pt }
\makeatother

\title{MapReduce Framework\\Final Report}

\author{
Derek Tzeng\\
Carnegie Mellon University\\
Pittsburgh, PA 15213 \\
\texttt{dtzeng@andrew.cmu.edu} \\
\And
Yiming Zong\\
Carnegie Mellon University\\
Pittsburgh, PA 15213 \\
\texttt{yzong@cmu.edu} \\
}

% The \author macro works with any number of authors. There are two commands
% used to separate the names and addresses of multiple authors: \And and \AND.
%
% Using \And between authors leaves it to \LaTeX{} to determine where to break
% the lines. Using \AND forces a linebreak at that point. So, if \LaTeX{}
% puts 3 of 4 authors names on the first line, and the last on the second
% line, try using \AND instead of \And before the third author name.

\newcommand{\fix}{\marginpar{FIX}}
\newcommand{\new}{\marginpar{NEW}}

\nipsfinalcopy % Uncomment for camera-ready version

\begin{document}


\maketitle

\begin{abstract}
This report reflects our final progress on MapReduce Framework project for 15-440, Fall 2014. We will first give an overview of the framework and discuss its components one by one. Then, we will describe the lifecycle of a MapReduce job and demonstrate the communication protocol between MapReduce components. Also, we will outline how to build and test the current framework from source, and include a Developer's Guide for developing MapReduce Applications based on the framework. Eventually, we will survey some additional unimplemented features that would be desirable for a commercial package.
\end{abstract}

\section{Overview of MapReduce Framework}

\par\qquad MapReduce is a highly scalable and parallel computing model that allows application programmers to specify \emph{Map} and \emph{Reduce} methods on a distributed data set in order to complete computational tasks. According to our design, nodes on the framework have one of the three following roles: \emph{Master}, \emph{Worker}, and the \emph{Users}.
\par\qquad Essentially, \emph{Master} is the coordinating node that does not store any data or perform any computational tasks but merely work on administrative tasks like maintaining a global Job Queue and a DFS Lookup Table. On the other hand, a \emph{Worker} receives and performs tasks pushed from \emph{Master}, and also holds partitions of data on the Distributed File System. Last but not least, a \emph{User} node is a designated client of the MapReduce framework, so where an end-user is able to push data on to the DFS and execute a MapReduce job based on the data.
\par\qquad Next section contains more detailed specifications of the three components.

\section{Framework Components}

\subsection{MapReduce/DFS Master}

\par

\par Meanwhile, it is important to notice that our Master is a \emph{single point of failure} in our framework, because once it fails, all information related to Jobs, Tasks, and Distributed Files are lost, and recovery is impossible because we also lose the routing rule of partial results to the requesting client.

\subsection{MapReduce/DFS Workers}

\par\qquad They carry out all the work.

\subsection{MapReduce Users}

\par\qquad In our design, MapReduce users are specifically designated nodes that are allowed to interact with the MapReduce cluster by contacting the Master node. 

\section{Lifecycle of MapReduce Job}
Here is what happens when a MapReduce job is executed:
\subsection*{Step One: User Load Data onto DFS}
\subsection*{Step Two: User Execute Task from User CLI}
\subsection*{Step Three: Mappers Feed Results to Sorters}
\subsection*{Step Four: Sorters Feed Results to Reducers}
\subsection*{Step Five: Reducers Return Results Back to Client}

\section{Communication Protocols}
\par\qquad As seen in the previous section, the key of the MapReduce framework is its management of concurrent requests, especially at the Master node. The following two sub-section will survey the network interactions between MapReduce Master node and two other components. 

\subsection{Communication Between MapReduce User \& MapReduce Master}

\par\qquad By discussion in \emph{Section One}, (What master does). Following are the communication details of each method:

\subsection{Communication Protocol: MapReduce Master \& MapReduce Workers}

\subsection{Failure Handling}
\par\qquad Due to the inherent irreliability of a network, it is not unlikely that some request packages fail to go through temporarily. Also, because potentially a large number of machines can be involved in the framework, it is fairly possible that some nodes goes down while running a task. Without proper failure handling, that faulty node can be a \emph{single point of failure (SPOF)}, which is what we want to avoid in a distributed system. Therefore, we addopted the following three approaches to handle failures:
\begin{easylist}[itemize]
    & Periodic Heartbeat: The master node sends periodic \texttt{ping} requests to all workers. If after a certain amount of time the worker still has not responded, we mark the worker as ``faulty'', and migrate the tasks already assigned to it to other nodes (see \emph{Task Migration} below);
    & Task Retry: When a task fails on a node for some reason, we allow the task to be re-routed to another node for re-try before marking the task as failed. With this measure, we avoid the case when a task and a work don't work well -- for example, when the underlying file system of that worker node has a bad sector;
    & Task Migration: After we have determined that a node is faulty, we re-distribute all tasks running on that node to other available nodes. This this case, we do not need to re-start the entire Job only becaues of one faulty worker node.
\end{easylist}



\section{Special Considerations}
\par\qquad Following are speical features that the authors thinks are useful for application programmers and are beyond the request of the project specification:

\begin{itemize}
    \item{\emph{Command Line Utility at RMI Servers and Users:}} 
    \item{\emph{Elegant Error Catching:}}
    \item{\emph{MapReduce Elegant Shutdown:}}
    \item{\emph{Cluster Failure Checking:}}
    \item{\emph{User node specified:}}
\end{itemize}

\section{Developer's Guide -- Using MapReduce Framework}

\par\qquad Since the MapReduce Framework is to be used by application developers, a pre-compiled Javadoc of the project can be found in the \texttt{doc/} directory under the project root. Users can simply open \texttt{doc/index.html} and access the documentation of the entire code base including the API usage guides. The mechanism of MapReduce Framework has been explained in previous sections.

\par\qquad In order to fit a Java Object into the RMI Framework, say, \texttt{WordCount}, the application programmer should create three classes: \texttt{WordCountMapper} that extends the base \texttt{Mapper} class, and similarly \texttt{WordCountSorter} and \texttt{WordCountReducer}. Then, the developer should put the classes under \texttt{mapr/examples} directory, such that MapReduce users are able to execute job based on our definition of \emph{Mapper}, \emph{Sorter}, and \emph{Reducer}.

\par\qquad In general, application developers are strongly recommended to look at the sample applications in the Package \texttt{mapr.examples}, and follow the existing code. We have provided the sample code for WordCount and Grep, and other MapReduce applications can be built similarly.

\section{Dependencies, Building, and Testing}

\par\qquad The \texttt{dtzeng.yzong} handin directory will contain two sub-directories, i.e. \texttt{reports} and \texttt{RMIFramework}. In the former directory you can find this report file, and in the latter directory is the clean source code for the project.

\subsection{Dependencies}

\par\qquad This project does not have any external dependencies, and can be compiled with standard Java libraries. To build the project from scratch without using IDE, see the following section.

\subsection{Building Instructions}

\par\qquad To build the project from scratch, use the "build.sh" script provided in the handin directory at "/afs/andrew/course/15/440-f14/handin/lab3/dtzeng-yzong.x".

\begin{Verbatim}[commandchars=\#\[\]]
gkesden@ghc11 dtzeng-yzong.x$ #userinput[chmod +x build.sh]
gkesden@ghc11 dtzeng-yzong.x$ #userinput[./build.sh]

\end{Verbatim}

\par\qquad This will create a jar file called "MapReduce.jar".  You may copy this jar to any desired directory for testing below.

\subsection{Testing Instructions}

\par\qquad The following test routine is designed to survey most functionalities of the MapReduce framework. Feel free to experiment with different commands in MapReduce Master and MapReduce User command-line interfaces.  In our example, our facility has 1 master and 3 workers, with 2 users connected.  Let GHC_M be the host of the master process, GHC_W1/2/3 the workers, and GHC_U1/2 the users.  more blah........................

To start up facility, do the following:
1. Start up master
\begin{Verbatim}[fontsize=\small, xleftmargin=-.5in,commandchars=\#\[\]]
gkesden@#comment[GHC_M] dtzeng-yzong.x$ #userinput[java -cp MapReduce.jar mapr.master.MasterCoordinator mapr.properties]
\end{Verbatim}
\\
2. On each of the workers
\begin{Verbatim}[fontsize=\small, xleftmargin=-.5in,commandchars=\#\[\]]
gkesden@#comment[GHC_W?] dtzeng-yzong.x$ #userinput[java -cp MapReduce.jar mapr.worker.WorkerCoordinator mapr.properties worker?]
\end{Verbatim}
\\
Now that the facility has started up, the command line on the master should appear.  Make sure to connect at least 1 worker to the master before the specified timeout, otherwise the facility will shutdown automatically.  With the facility started up, you can now start the 2 users.\\
3. On each of the users
\begin{Verbatim}[fontsize=\small, xleftmargin=-.5in,commandchars=\#\[\]]
gkesden@#comment[GHC_U?] dtzeng-yzong.x$ #userinput[java -cp MapReduce.jar mapr.user.UserCoordinator mapr.properties user?]
\end{Verbatim}
\\
So, now we have everything up and running.  On the master, you can use the "workers" and "users" commands to see the status of the workers and users.  Now, let's upload our example files.  You can do this on any user.  This demonstrates on user1.
\begin{Verbatim}[fontsize=\small, xleftmargin=-.5in,commandchars=\#\[\]]
user1@9876 > #userinput[upload large_count.log]
File replicated successfully.
user1@9876 > #userinput[upload large_grep.log]
File replicated successfully.
\end{Verbatim}
\\
Now, when you run the "files" command on either user or master, you can see the information of the files on the DFS, and where the replications are located.  Let's now start a new word count and grep job.
\begin{Verbatim}[fontsize=\small, xleftmargin=-.5in,commandchars=\#\[\]]
user1@9876 > #userinput[count large_count.log 0 99999 ctr_out]
Job started with ID 0
user1@9876 > #userinput[grep large_grep.log 0 99999 grep_out tzeng]
Job started with ID 1
\end{Verbatim}
You can now use the "jobs" command on either user or master, to see the status of the job.  To see the individual tasks, use "tasks" on the master machine only.  When everything is finished, you can see the results of each job in the user's specified DFS directory, or in our case "worker1-dfs-root".  Now let's see what happens when a worker fails.  Start another job, and kill a worker.
\begin{Verbatim}[fontsize=\small, xleftmargin=-.5in,commandchars=\#\[\]]
user1@9876 > #userinput[count large_count.log 0 99999 ctr_out]
Job started with ID 2
	#comment[*** Kill a worker, in this example worker 1 ***]
gkesden@#comment[GHC_W1] dtzeng-yzong.x$ #userinput[java -cp MapReduce.jar mapr.worker.WorkerCoordinator mapr.properties worker1]
Successful connection to facility master established.
\^C gkesden@#comment[GHC_W1] dtzeng-yzong.x$
\end{Verbatim}
When you run the "jobs" command again, you will see that the status for job 2, becomes "RESTARTED AS JOB 3".  Job 3 will now finish what job 2 requested.  If you try the "files" command, you can see that the replicas on the killed worker (worker1) are removed.\\
Lastly, to shut down the facility, use the "shutdown" command on the master CLI.  The users are not included in the facility, so another interesting feature is the "reconnect" command on the user.  If the facility is started up again (masters and workers), the user can simply try to reconnect, and it will automatically be added back into the master's state.

\section{Further Work \& Enhancements}

Due to time constraints, although the final product is fully functional, it still lacks many features that are desired in commercial packages (like \emph{Hadoop}), and their implementation difficulties also vary. This section will mention a selected few and discuss the difficulty for implementing them.

\subsection{MapReduce Job Priority}


\subsection{Fully Support Hadoop-wise Developer API}


\subsection{Elegant RMI Exception Reporting}


\subsection{Daemonize MapReduce Framework}

\urlstyle{rm}

\end{document}
